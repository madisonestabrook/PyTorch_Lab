
% Default to the notebook output style

    


% Inherit from the specified cell style.




    
\documentclass[11pt]{article}

    
    
    \usepackage[T1]{fontenc}
    % Nicer default font (+ math font) than Computer Modern for most use cases
    \usepackage{mathpazo}

    % Basic figure setup, for now with no caption control since it's done
    % automatically by Pandoc (which extracts ![](path) syntax from Markdown).
    \usepackage{graphicx}
    % We will generate all images so they have a width \maxwidth. This means
    % that they will get their normal width if they fit onto the page, but
    % are scaled down if they would overflow the margins.
    \makeatletter
    \def\maxwidth{\ifdim\Gin@nat@width>\linewidth\linewidth
    \else\Gin@nat@width\fi}
    \makeatother
    \let\Oldincludegraphics\includegraphics
    % Set max figure width to be 80% of text width, for now hardcoded.
    \renewcommand{\includegraphics}[1]{\Oldincludegraphics[width=.8\maxwidth]{#1}}
    % Ensure that by default, figures have no caption (until we provide a
    % proper Figure object with a Caption API and a way to capture that
    % in the conversion process - todo).
    \usepackage{caption}
    \DeclareCaptionLabelFormat{nolabel}{}
    \captionsetup{labelformat=nolabel}

    \usepackage{adjustbox} % Used to constrain images to a maximum size 
    \usepackage{xcolor} % Allow colors to be defined
    \usepackage{enumerate} % Needed for markdown enumerations to work
    \usepackage{geometry} % Used to adjust the document margins
    \usepackage{amsmath} % Equations
    \usepackage{amssymb} % Equations
    \usepackage{textcomp} % defines textquotesingle
    % Hack from http://tex.stackexchange.com/a/47451/13684:
    \AtBeginDocument{%
        \def\PYZsq{\textquotesingle}% Upright quotes in Pygmentized code
    }
    \usepackage{upquote} % Upright quotes for verbatim code
    \usepackage{eurosym} % defines \euro
    \usepackage[mathletters]{ucs} % Extended unicode (utf-8) support
    \usepackage[utf8x]{inputenc} % Allow utf-8 characters in the tex document
    \usepackage{fancyvrb} % verbatim replacement that allows latex
    \usepackage{grffile} % extends the file name processing of package graphics 
                         % to support a larger range 
    % The hyperref package gives us a pdf with properly built
    % internal navigation ('pdf bookmarks' for the table of contents,
    % internal cross-reference links, web links for URLs, etc.)
    \usepackage{hyperref}
    \usepackage{longtable} % longtable support required by pandoc >1.10
    \usepackage{booktabs}  % table support for pandoc > 1.12.2
    \usepackage[inline]{enumitem} % IRkernel/repr support (it uses the enumerate* environment)
    \usepackage[normalem]{ulem} % ulem is needed to support strikethroughs (\sout)
                                % normalem makes italics be italics, not underlines
    

    
    
    % Colors for the hyperref package
    \definecolor{urlcolor}{rgb}{0,.145,.698}
    \definecolor{linkcolor}{rgb}{.71,0.21,0.01}
    \definecolor{citecolor}{rgb}{.12,.54,.11}

    % ANSI colors
    \definecolor{ansi-black}{HTML}{3E424D}
    \definecolor{ansi-black-intense}{HTML}{282C36}
    \definecolor{ansi-red}{HTML}{E75C58}
    \definecolor{ansi-red-intense}{HTML}{B22B31}
    \definecolor{ansi-green}{HTML}{00A250}
    \definecolor{ansi-green-intense}{HTML}{007427}
    \definecolor{ansi-yellow}{HTML}{DDB62B}
    \definecolor{ansi-yellow-intense}{HTML}{B27D12}
    \definecolor{ansi-blue}{HTML}{208FFB}
    \definecolor{ansi-blue-intense}{HTML}{0065CA}
    \definecolor{ansi-magenta}{HTML}{D160C4}
    \definecolor{ansi-magenta-intense}{HTML}{A03196}
    \definecolor{ansi-cyan}{HTML}{60C6C8}
    \definecolor{ansi-cyan-intense}{HTML}{258F8F}
    \definecolor{ansi-white}{HTML}{C5C1B4}
    \definecolor{ansi-white-intense}{HTML}{A1A6B2}

    % commands and environments needed by pandoc snippets
    % extracted from the output of `pandoc -s`
    \providecommand{\tightlist}{%
      \setlength{\itemsep}{0pt}\setlength{\parskip}{0pt}}
    \DefineVerbatimEnvironment{Highlighting}{Verbatim}{commandchars=\\\{\}}
    % Add ',fontsize=\small' for more characters per line
    \newenvironment{Shaded}{}{}
    \newcommand{\KeywordTok}[1]{\textcolor[rgb]{0.00,0.44,0.13}{\textbf{{#1}}}}
    \newcommand{\DataTypeTok}[1]{\textcolor[rgb]{0.56,0.13,0.00}{{#1}}}
    \newcommand{\DecValTok}[1]{\textcolor[rgb]{0.25,0.63,0.44}{{#1}}}
    \newcommand{\BaseNTok}[1]{\textcolor[rgb]{0.25,0.63,0.44}{{#1}}}
    \newcommand{\FloatTok}[1]{\textcolor[rgb]{0.25,0.63,0.44}{{#1}}}
    \newcommand{\CharTok}[1]{\textcolor[rgb]{0.25,0.44,0.63}{{#1}}}
    \newcommand{\StringTok}[1]{\textcolor[rgb]{0.25,0.44,0.63}{{#1}}}
    \newcommand{\CommentTok}[1]{\textcolor[rgb]{0.38,0.63,0.69}{\textit{{#1}}}}
    \newcommand{\OtherTok}[1]{\textcolor[rgb]{0.00,0.44,0.13}{{#1}}}
    \newcommand{\AlertTok}[1]{\textcolor[rgb]{1.00,0.00,0.00}{\textbf{{#1}}}}
    \newcommand{\FunctionTok}[1]{\textcolor[rgb]{0.02,0.16,0.49}{{#1}}}
    \newcommand{\RegionMarkerTok}[1]{{#1}}
    \newcommand{\ErrorTok}[1]{\textcolor[rgb]{1.00,0.00,0.00}{\textbf{{#1}}}}
    \newcommand{\NormalTok}[1]{{#1}}
    
    % Additional commands for more recent versions of Pandoc
    \newcommand{\ConstantTok}[1]{\textcolor[rgb]{0.53,0.00,0.00}{{#1}}}
    \newcommand{\SpecialCharTok}[1]{\textcolor[rgb]{0.25,0.44,0.63}{{#1}}}
    \newcommand{\VerbatimStringTok}[1]{\textcolor[rgb]{0.25,0.44,0.63}{{#1}}}
    \newcommand{\SpecialStringTok}[1]{\textcolor[rgb]{0.73,0.40,0.53}{{#1}}}
    \newcommand{\ImportTok}[1]{{#1}}
    \newcommand{\DocumentationTok}[1]{\textcolor[rgb]{0.73,0.13,0.13}{\textit{{#1}}}}
    \newcommand{\AnnotationTok}[1]{\textcolor[rgb]{0.38,0.63,0.69}{\textbf{\textit{{#1}}}}}
    \newcommand{\CommentVarTok}[1]{\textcolor[rgb]{0.38,0.63,0.69}{\textbf{\textit{{#1}}}}}
    \newcommand{\VariableTok}[1]{\textcolor[rgb]{0.10,0.09,0.49}{{#1}}}
    \newcommand{\ControlFlowTok}[1]{\textcolor[rgb]{0.00,0.44,0.13}{\textbf{{#1}}}}
    \newcommand{\OperatorTok}[1]{\textcolor[rgb]{0.40,0.40,0.40}{{#1}}}
    \newcommand{\BuiltInTok}[1]{{#1}}
    \newcommand{\ExtensionTok}[1]{{#1}}
    \newcommand{\PreprocessorTok}[1]{\textcolor[rgb]{0.74,0.48,0.00}{{#1}}}
    \newcommand{\AttributeTok}[1]{\textcolor[rgb]{0.49,0.56,0.16}{{#1}}}
    \newcommand{\InformationTok}[1]{\textcolor[rgb]{0.38,0.63,0.69}{\textbf{\textit{{#1}}}}}
    \newcommand{\WarningTok}[1]{\textcolor[rgb]{0.38,0.63,0.69}{\textbf{\textit{{#1}}}}}
    
    
    % Define a nice break command that doesn't care if a line doesn't already
    % exist.
    \def\br{\hspace*{\fill} \\* }
    % Math Jax compatability definitions
    \def\gt{>}
    \def\lt{<}
    % Document parameters
    \title{Image Classifier Project}
    
    
    

    % Pygments definitions
    
\makeatletter
\def\PY@reset{\let\PY@it=\relax \let\PY@bf=\relax%
    \let\PY@ul=\relax \let\PY@tc=\relax%
    \let\PY@bc=\relax \let\PY@ff=\relax}
\def\PY@tok#1{\csname PY@tok@#1\endcsname}
\def\PY@toks#1+{\ifx\relax#1\empty\else%
    \PY@tok{#1}\expandafter\PY@toks\fi}
\def\PY@do#1{\PY@bc{\PY@tc{\PY@ul{%
    \PY@it{\PY@bf{\PY@ff{#1}}}}}}}
\def\PY#1#2{\PY@reset\PY@toks#1+\relax+\PY@do{#2}}

\expandafter\def\csname PY@tok@w\endcsname{\def\PY@tc##1{\textcolor[rgb]{0.73,0.73,0.73}{##1}}}
\expandafter\def\csname PY@tok@c\endcsname{\let\PY@it=\textit\def\PY@tc##1{\textcolor[rgb]{0.25,0.50,0.50}{##1}}}
\expandafter\def\csname PY@tok@cp\endcsname{\def\PY@tc##1{\textcolor[rgb]{0.74,0.48,0.00}{##1}}}
\expandafter\def\csname PY@tok@k\endcsname{\let\PY@bf=\textbf\def\PY@tc##1{\textcolor[rgb]{0.00,0.50,0.00}{##1}}}
\expandafter\def\csname PY@tok@kp\endcsname{\def\PY@tc##1{\textcolor[rgb]{0.00,0.50,0.00}{##1}}}
\expandafter\def\csname PY@tok@kt\endcsname{\def\PY@tc##1{\textcolor[rgb]{0.69,0.00,0.25}{##1}}}
\expandafter\def\csname PY@tok@o\endcsname{\def\PY@tc##1{\textcolor[rgb]{0.40,0.40,0.40}{##1}}}
\expandafter\def\csname PY@tok@ow\endcsname{\let\PY@bf=\textbf\def\PY@tc##1{\textcolor[rgb]{0.67,0.13,1.00}{##1}}}
\expandafter\def\csname PY@tok@nb\endcsname{\def\PY@tc##1{\textcolor[rgb]{0.00,0.50,0.00}{##1}}}
\expandafter\def\csname PY@tok@nf\endcsname{\def\PY@tc##1{\textcolor[rgb]{0.00,0.00,1.00}{##1}}}
\expandafter\def\csname PY@tok@nc\endcsname{\let\PY@bf=\textbf\def\PY@tc##1{\textcolor[rgb]{0.00,0.00,1.00}{##1}}}
\expandafter\def\csname PY@tok@nn\endcsname{\let\PY@bf=\textbf\def\PY@tc##1{\textcolor[rgb]{0.00,0.00,1.00}{##1}}}
\expandafter\def\csname PY@tok@ne\endcsname{\let\PY@bf=\textbf\def\PY@tc##1{\textcolor[rgb]{0.82,0.25,0.23}{##1}}}
\expandafter\def\csname PY@tok@nv\endcsname{\def\PY@tc##1{\textcolor[rgb]{0.10,0.09,0.49}{##1}}}
\expandafter\def\csname PY@tok@no\endcsname{\def\PY@tc##1{\textcolor[rgb]{0.53,0.00,0.00}{##1}}}
\expandafter\def\csname PY@tok@nl\endcsname{\def\PY@tc##1{\textcolor[rgb]{0.63,0.63,0.00}{##1}}}
\expandafter\def\csname PY@tok@ni\endcsname{\let\PY@bf=\textbf\def\PY@tc##1{\textcolor[rgb]{0.60,0.60,0.60}{##1}}}
\expandafter\def\csname PY@tok@na\endcsname{\def\PY@tc##1{\textcolor[rgb]{0.49,0.56,0.16}{##1}}}
\expandafter\def\csname PY@tok@nt\endcsname{\let\PY@bf=\textbf\def\PY@tc##1{\textcolor[rgb]{0.00,0.50,0.00}{##1}}}
\expandafter\def\csname PY@tok@nd\endcsname{\def\PY@tc##1{\textcolor[rgb]{0.67,0.13,1.00}{##1}}}
\expandafter\def\csname PY@tok@s\endcsname{\def\PY@tc##1{\textcolor[rgb]{0.73,0.13,0.13}{##1}}}
\expandafter\def\csname PY@tok@sd\endcsname{\let\PY@it=\textit\def\PY@tc##1{\textcolor[rgb]{0.73,0.13,0.13}{##1}}}
\expandafter\def\csname PY@tok@si\endcsname{\let\PY@bf=\textbf\def\PY@tc##1{\textcolor[rgb]{0.73,0.40,0.53}{##1}}}
\expandafter\def\csname PY@tok@se\endcsname{\let\PY@bf=\textbf\def\PY@tc##1{\textcolor[rgb]{0.73,0.40,0.13}{##1}}}
\expandafter\def\csname PY@tok@sr\endcsname{\def\PY@tc##1{\textcolor[rgb]{0.73,0.40,0.53}{##1}}}
\expandafter\def\csname PY@tok@ss\endcsname{\def\PY@tc##1{\textcolor[rgb]{0.10,0.09,0.49}{##1}}}
\expandafter\def\csname PY@tok@sx\endcsname{\def\PY@tc##1{\textcolor[rgb]{0.00,0.50,0.00}{##1}}}
\expandafter\def\csname PY@tok@m\endcsname{\def\PY@tc##1{\textcolor[rgb]{0.40,0.40,0.40}{##1}}}
\expandafter\def\csname PY@tok@gh\endcsname{\let\PY@bf=\textbf\def\PY@tc##1{\textcolor[rgb]{0.00,0.00,0.50}{##1}}}
\expandafter\def\csname PY@tok@gu\endcsname{\let\PY@bf=\textbf\def\PY@tc##1{\textcolor[rgb]{0.50,0.00,0.50}{##1}}}
\expandafter\def\csname PY@tok@gd\endcsname{\def\PY@tc##1{\textcolor[rgb]{0.63,0.00,0.00}{##1}}}
\expandafter\def\csname PY@tok@gi\endcsname{\def\PY@tc##1{\textcolor[rgb]{0.00,0.63,0.00}{##1}}}
\expandafter\def\csname PY@tok@gr\endcsname{\def\PY@tc##1{\textcolor[rgb]{1.00,0.00,0.00}{##1}}}
\expandafter\def\csname PY@tok@ge\endcsname{\let\PY@it=\textit}
\expandafter\def\csname PY@tok@gs\endcsname{\let\PY@bf=\textbf}
\expandafter\def\csname PY@tok@gp\endcsname{\let\PY@bf=\textbf\def\PY@tc##1{\textcolor[rgb]{0.00,0.00,0.50}{##1}}}
\expandafter\def\csname PY@tok@go\endcsname{\def\PY@tc##1{\textcolor[rgb]{0.53,0.53,0.53}{##1}}}
\expandafter\def\csname PY@tok@gt\endcsname{\def\PY@tc##1{\textcolor[rgb]{0.00,0.27,0.87}{##1}}}
\expandafter\def\csname PY@tok@err\endcsname{\def\PY@bc##1{\setlength{\fboxsep}{0pt}\fcolorbox[rgb]{1.00,0.00,0.00}{1,1,1}{\strut ##1}}}
\expandafter\def\csname PY@tok@kc\endcsname{\let\PY@bf=\textbf\def\PY@tc##1{\textcolor[rgb]{0.00,0.50,0.00}{##1}}}
\expandafter\def\csname PY@tok@kd\endcsname{\let\PY@bf=\textbf\def\PY@tc##1{\textcolor[rgb]{0.00,0.50,0.00}{##1}}}
\expandafter\def\csname PY@tok@kn\endcsname{\let\PY@bf=\textbf\def\PY@tc##1{\textcolor[rgb]{0.00,0.50,0.00}{##1}}}
\expandafter\def\csname PY@tok@kr\endcsname{\let\PY@bf=\textbf\def\PY@tc##1{\textcolor[rgb]{0.00,0.50,0.00}{##1}}}
\expandafter\def\csname PY@tok@bp\endcsname{\def\PY@tc##1{\textcolor[rgb]{0.00,0.50,0.00}{##1}}}
\expandafter\def\csname PY@tok@fm\endcsname{\def\PY@tc##1{\textcolor[rgb]{0.00,0.00,1.00}{##1}}}
\expandafter\def\csname PY@tok@vc\endcsname{\def\PY@tc##1{\textcolor[rgb]{0.10,0.09,0.49}{##1}}}
\expandafter\def\csname PY@tok@vg\endcsname{\def\PY@tc##1{\textcolor[rgb]{0.10,0.09,0.49}{##1}}}
\expandafter\def\csname PY@tok@vi\endcsname{\def\PY@tc##1{\textcolor[rgb]{0.10,0.09,0.49}{##1}}}
\expandafter\def\csname PY@tok@vm\endcsname{\def\PY@tc##1{\textcolor[rgb]{0.10,0.09,0.49}{##1}}}
\expandafter\def\csname PY@tok@sa\endcsname{\def\PY@tc##1{\textcolor[rgb]{0.73,0.13,0.13}{##1}}}
\expandafter\def\csname PY@tok@sb\endcsname{\def\PY@tc##1{\textcolor[rgb]{0.73,0.13,0.13}{##1}}}
\expandafter\def\csname PY@tok@sc\endcsname{\def\PY@tc##1{\textcolor[rgb]{0.73,0.13,0.13}{##1}}}
\expandafter\def\csname PY@tok@dl\endcsname{\def\PY@tc##1{\textcolor[rgb]{0.73,0.13,0.13}{##1}}}
\expandafter\def\csname PY@tok@s2\endcsname{\def\PY@tc##1{\textcolor[rgb]{0.73,0.13,0.13}{##1}}}
\expandafter\def\csname PY@tok@sh\endcsname{\def\PY@tc##1{\textcolor[rgb]{0.73,0.13,0.13}{##1}}}
\expandafter\def\csname PY@tok@s1\endcsname{\def\PY@tc##1{\textcolor[rgb]{0.73,0.13,0.13}{##1}}}
\expandafter\def\csname PY@tok@mb\endcsname{\def\PY@tc##1{\textcolor[rgb]{0.40,0.40,0.40}{##1}}}
\expandafter\def\csname PY@tok@mf\endcsname{\def\PY@tc##1{\textcolor[rgb]{0.40,0.40,0.40}{##1}}}
\expandafter\def\csname PY@tok@mh\endcsname{\def\PY@tc##1{\textcolor[rgb]{0.40,0.40,0.40}{##1}}}
\expandafter\def\csname PY@tok@mi\endcsname{\def\PY@tc##1{\textcolor[rgb]{0.40,0.40,0.40}{##1}}}
\expandafter\def\csname PY@tok@il\endcsname{\def\PY@tc##1{\textcolor[rgb]{0.40,0.40,0.40}{##1}}}
\expandafter\def\csname PY@tok@mo\endcsname{\def\PY@tc##1{\textcolor[rgb]{0.40,0.40,0.40}{##1}}}
\expandafter\def\csname PY@tok@ch\endcsname{\let\PY@it=\textit\def\PY@tc##1{\textcolor[rgb]{0.25,0.50,0.50}{##1}}}
\expandafter\def\csname PY@tok@cm\endcsname{\let\PY@it=\textit\def\PY@tc##1{\textcolor[rgb]{0.25,0.50,0.50}{##1}}}
\expandafter\def\csname PY@tok@cpf\endcsname{\let\PY@it=\textit\def\PY@tc##1{\textcolor[rgb]{0.25,0.50,0.50}{##1}}}
\expandafter\def\csname PY@tok@c1\endcsname{\let\PY@it=\textit\def\PY@tc##1{\textcolor[rgb]{0.25,0.50,0.50}{##1}}}
\expandafter\def\csname PY@tok@cs\endcsname{\let\PY@it=\textit\def\PY@tc##1{\textcolor[rgb]{0.25,0.50,0.50}{##1}}}

\def\PYZbs{\char`\\}
\def\PYZus{\char`\_}
\def\PYZob{\char`\{}
\def\PYZcb{\char`\}}
\def\PYZca{\char`\^}
\def\PYZam{\char`\&}
\def\PYZlt{\char`\<}
\def\PYZgt{\char`\>}
\def\PYZsh{\char`\#}
\def\PYZpc{\char`\%}
\def\PYZdl{\char`\$}
\def\PYZhy{\char`\-}
\def\PYZsq{\char`\'}
\def\PYZdq{\char`\"}
\def\PYZti{\char`\~}
% for compatibility with earlier versions
\def\PYZat{@}
\def\PYZlb{[}
\def\PYZrb{]}
\makeatother


    % Exact colors from NB
    \definecolor{incolor}{rgb}{0.0, 0.0, 0.5}
    \definecolor{outcolor}{rgb}{0.545, 0.0, 0.0}



    
    % Prevent overflowing lines due to hard-to-break entities
    \sloppy 
    % Setup hyperref package
    \hypersetup{
      breaklinks=true,  % so long urls are correctly broken across lines
      colorlinks=true,
      urlcolor=urlcolor,
      linkcolor=linkcolor,
      citecolor=citecolor,
      }
    % Slightly bigger margins than the latex defaults
    
    \geometry{verbose,tmargin=1in,bmargin=1in,lmargin=1in,rmargin=1in}
    
    

    \begin{document}
    
    
    \maketitle
    
    

    
    \section{Developing an AI
application}\label{developing-an-ai-application}

Going forward, AI algorithms will be incorporated into more and more
everyday applications. For example, you might want to include an image
classifier in a smart phone app. To do this, you'd use a deep learning
model trained on hundreds of thousands of images as part of the overall
application architecture. A large part of software development in the
future will be using these types of models as common parts of
applications.

In this project, you'll train an image classifier to recognize different
species of flowers. You can imagine using something like this in a phone
app that tells you the name of the flower your camera is looking at. In
practice you'd train this classifier, then export it for use in your
application. We'll be using
\href{http://www.robots.ox.ac.uk/~vgg/data/flowers/102/index.html}{this
dataset} of 102 flower categories, you can see a few examples below.

The project is broken down into multiple steps:

\begin{itemize}
\tightlist
\item
  Load and preprocess the image dataset
\item
  Train the image classifier on your dataset
\item
  Use the trained classifier to predict image content
\end{itemize}

We'll lead you through each part which you'll implement in Python.

When you've completed this project, you'll have an application that can
be trained on any set of labeled images. Here your network will be
learning about flowers and end up as a command line application. But,
what you do with your new skills depends on your imagination and effort
in building a dataset. For example, imagine an app where you take a
picture of a car, it tells you what the make and model is, then looks up
information about it. Go build your own dataset and make something new.

First up is importing the packages you'll need. It's good practice to
keep all the imports at the beginning of your code. As you work through
this notebook and find you need to import a package, make sure to add
the import up here.

    \begin{Verbatim}[commandchars=\\\{\}]
{\color{incolor}In [{\color{incolor}2}]:} \PY{c+c1}{\PYZsh{} Imports here]}
        \PY{o}{\PYZpc{}}\PY{k}{matplotlib} inline
        \PY{o}{\PYZpc{}}\PY{k}{config} InlineBackend.figure\PYZus{}format = \PYZsq{}retina\PYZsq{}
        \PY{k+kn}{import} \PY{n+nn}{matplotlib}\PY{n+nn}{.}\PY{n+nn}{pyplot} \PY{k}{as} \PY{n+nn}{plt}
        \PY{k+kn}{import} \PY{n+nn}{numpy} \PY{k}{as} \PY{n+nn}{np}
        \PY{k+kn}{import} \PY{n+nn}{PIL}
        \PY{k+kn}{from} \PY{n+nn}{PIL} \PY{k}{import} \PY{n}{Image}
        
        \PY{k+kn}{import} \PY{n+nn}{torch}
        \PY{k+kn}{import} \PY{n+nn}{torchvision}
        \PY{k+kn}{from} \PY{n+nn}{torch} \PY{k}{import} \PY{n}{nn} 
        \PY{k+kn}{import} \PY{n+nn}{torch}\PY{n+nn}{.}\PY{n+nn}{nn}\PY{n+nn}{.}\PY{n+nn}{functional} \PY{k}{as} \PY{n+nn}{F}
        \PY{k+kn}{from} \PY{n+nn}{torch} \PY{k}{import} \PY{n}{optim} 
        \PY{k+kn}{from} \PY{n+nn}{torchvision} \PY{k}{import} \PY{n}{datasets}\PY{p}{,} \PY{n}{transforms}\PY{p}{,} \PY{n}{models}
\end{Verbatim}


    \subsection{Load the data}\label{load-the-data}

Here you'll use \texttt{torchvision} to load the data
(\href{http://pytorch.org/docs/0.3.0/torchvision/index.html}{documentation}).
You can
\href{https://s3.amazonaws.com/content.udacity-data.com/courses/nd188/flower_data.zip}{download
the data here}. The dataset is split into two parts, training and
validation. For the training, you'll want to apply transformations such
as random scaling, cropping, and flipping. This will help the network
generalize leading to better performance. If you use a pre-trained
network, you'll also need to make sure the input data is resized to
224x224 pixels as required by the networks.

The validation set is used to measure the model's performance on data it
hasn't seen yet. For this you don't want any scaling or rotation
transformations, but you'll need to resize then crop the images to the
appropriate size.

The pre-trained networks available from \texttt{torchvision} were
trained on the ImageNet dataset where each color channel was normalized
separately. For both sets you'll need to normalize the means and
standard deviations of the images to what the network expects. For the
means, it's \texttt{{[}0.485,\ 0.456,\ 0.406{]}} and for the standard
deviations \texttt{{[}0.229,\ 0.224,\ 0.225{]}}, calculated from the
ImageNet images. These values will shift each color channel to be
centered at 0 and range from -1 to 1.

    \begin{Verbatim}[commandchars=\\\{\}]
{\color{incolor}In [{\color{incolor}3}]:} \PY{n}{data\PYZus{}dir} \PY{o}{=} \PY{l+s+s1}{\PYZsq{}}\PY{l+s+s1}{flower\PYZus{}data}\PY{l+s+s1}{\PYZsq{}}
        \PY{n}{train\PYZus{}dir} \PY{o}{=} \PY{n}{data\PYZus{}dir} \PY{o}{+} \PY{l+s+s1}{\PYZsq{}}\PY{l+s+s1}{/train}\PY{l+s+s1}{\PYZsq{}}
        \PY{n}{valid\PYZus{}dir} \PY{o}{=} \PY{n}{data\PYZus{}dir} \PY{o}{+} \PY{l+s+s1}{\PYZsq{}}\PY{l+s+s1}{/valid}\PY{l+s+s1}{\PYZsq{}}
\end{Verbatim}


    \begin{Verbatim}[commandchars=\\\{\}]
{\color{incolor}In [{\color{incolor}4}]:} \PY{c+c1}{\PYZsh{} TODO: Define your transforms for the training and validation sets}
        \PY{n}{data\PYZus{}transforms} \PY{o}{=} \PY{n}{transforms}\PY{o}{.}\PY{n}{Compose}\PY{p}{(}\PY{p}{[}\PY{n}{transforms}\PY{o}{.}\PY{n}{RandomResizedCrop}\PY{p}{(}\PY{l+m+mi}{224}\PY{p}{)}\PY{p}{,}
                                              \PY{n}{transforms}\PY{o}{.}\PY{n}{ToTensor}\PY{p}{(}\PY{p}{)}\PY{p}{,} 
                                              \PY{n}{transforms}\PY{o}{.}\PY{n}{Normalize}\PY{p}{(}\PY{p}{[}\PY{l+m+mf}{0.485}\PY{p}{,} \PY{l+m+mf}{0.456}\PY{p}{,} \PY{l+m+mf}{0.406}\PY{p}{]}\PY{p}{,} 
                                              \PY{p}{[}\PY{l+m+mf}{0.229}\PY{p}{,} \PY{l+m+mf}{0.224}\PY{p}{,} \PY{l+m+mf}{0.225}\PY{p}{]}\PY{p}{)}\PY{p}{]}\PY{p}{)}
        
        \PY{n}{test\PYZus{}transforms} \PY{o}{=} \PY{n}{transforms}\PY{o}{.}\PY{n}{Compose}\PY{p}{(}\PY{p}{[}\PY{n}{transforms}\PY{o}{.}\PY{n}{RandomResizedCrop}\PY{p}{(}\PY{l+m+mi}{224}\PY{p}{)}\PY{p}{,}
                                              \PY{n}{transforms}\PY{o}{.}\PY{n}{ToTensor}\PY{p}{(}\PY{p}{)}\PY{p}{,} 
                                              \PY{n}{transforms}\PY{o}{.}\PY{n}{Normalize}\PY{p}{(}\PY{p}{[}\PY{l+m+mf}{0.485}\PY{p}{,} \PY{l+m+mf}{0.456}\PY{p}{,} \PY{l+m+mf}{0.406}\PY{p}{]}\PY{p}{,} 
                                              \PY{p}{[}\PY{l+m+mf}{0.229}\PY{p}{,} \PY{l+m+mf}{0.224}\PY{p}{,} \PY{l+m+mf}{0.225}\PY{p}{]}\PY{p}{)}\PY{p}{]}\PY{p}{)}
        
        \PY{c+c1}{\PYZsh{} TODO: Load the datasets with ImageFolder}
        \PY{n}{train\PYZus{}dataset} \PY{o}{=} \PY{n}{datasets}\PY{o}{.}\PY{n}{ImageFolder}\PY{p}{(}\PY{n}{train\PYZus{}dir}\PY{p}{,} \PY{n}{transform}\PY{o}{=}\PY{n}{data\PYZus{}transforms}\PY{p}{)}
        \PY{n}{test\PYZus{}dataset} \PY{o}{=} \PY{n}{datasets}\PY{o}{.}\PY{n}{ImageFolder}\PY{p}{(}\PY{n}{valid\PYZus{}dir} \PY{p}{,} \PY{n}{transform}\PY{o}{=}\PY{n}{test\PYZus{}transforms}\PY{p}{)}
        
        \PY{c+c1}{\PYZsh{} TODO: Using the image datasets and the trainforms, define the dataloaders}
        \PY{n}{trainloader} \PY{o}{=} \PY{n}{torch}\PY{o}{.}\PY{n}{utils}\PY{o}{.}\PY{n}{data}\PY{o}{.}\PY{n}{DataLoader}\PY{p}{(}\PY{n}{train\PYZus{}dataset}\PY{p}{,} \PY{n}{batch\PYZus{}size}\PY{o}{=}\PY{l+m+mi}{32}\PY{p}{,} \PY{n}{num\PYZus{}workers}\PY{o}{=}\PY{l+m+mi}{2}\PY{p}{,} \PY{n}{shuffle}\PY{o}{=}\PY{k+kc}{True}\PY{p}{)} 
        \PY{n}{testloader} \PY{o}{=} \PY{n}{torch}\PY{o}{.}\PY{n}{utils}\PY{o}{.}\PY{n}{data}\PY{o}{.}\PY{n}{DataLoader}\PY{p}{(}\PY{n}{test\PYZus{}dataset}\PY{p}{,} \PY{n}{batch\PYZus{}size}\PY{o}{=}\PY{l+m+mi}{32}\PY{p}{,} \PY{n}{num\PYZus{}workers}\PY{o}{=}\PY{l+m+mi}{2}\PY{p}{,} \PY{n}{shuffle}\PY{o}{=}\PY{k+kc}{True}\PY{p}{)} 
        \PY{c+c1}{\PYZsh{}image\PYZus{}datasets = [trainloader, testloader]}
\end{Verbatim}


    \subsubsection{Label mapping}\label{label-mapping}

You'll also need to load in a mapping from category label to category
name. You can find this in the file \texttt{cat\_to\_name.json}. It's a
JSON object which you can read in with the
\href{https://docs.python.org/2/library/json.html}{\texttt{json}
module}. This will give you a dictionary mapping the integer encoded
categories to the actual names of the flowers.

    \begin{Verbatim}[commandchars=\\\{\}]
{\color{incolor}In [{\color{incolor}5}]:} \PY{k+kn}{import} \PY{n+nn}{json}
        
        \PY{k}{with} \PY{n+nb}{open}\PY{p}{(}\PY{l+s+s1}{\PYZsq{}}\PY{l+s+s1}{cat\PYZus{}to\PYZus{}name.json}\PY{l+s+s1}{\PYZsq{}}\PY{p}{,} \PY{l+s+s1}{\PYZsq{}}\PY{l+s+s1}{r}\PY{l+s+s1}{\PYZsq{}}\PY{p}{)} \PY{k}{as} \PY{n}{f}\PY{p}{:}
            \PY{n}{cat\PYZus{}to\PYZus{}name} \PY{o}{=} \PY{n}{json}\PY{o}{.}\PY{n}{load}\PY{p}{(}\PY{n}{f}\PY{p}{)}
\end{Verbatim}


    \section{Building and training the
classifier}\label{building-and-training-the-classifier}

Now that the data is ready, it's time to build and train the classifier.
As usual, you should use one of the pretrained models from
\texttt{torchvision.models} to get the image features. Build and train a
new feed-forward classifier using those features.

We're going to leave this part up to you. If you want to talk through it
with someone, chat with your fellow students! You can also ask questions
on the forums or join the instructors in office hours.

Refer to \href{https://review.udacity.com/\#!/rubrics/1663/view}{the
rubric} for guidance on successfully completing this section. Things
you'll need to do:

\begin{itemize}
\tightlist
\item
  Load a
  \href{http://pytorch.org/docs/master/torchvision/models.html}{pre-trained
  network} (If you need a starting point, the VGG networks work great
  and are straightforward to use)
\item
  Define a new, untrained feed-forward network as a classifier, using
  ReLU activations and dropout
\item
  Train the classifier layers using backpropagation using the
  pre-trained network to get the features
\item
  Track the loss and accuracy on the validation set to determine the
  best hyperparameters
\end{itemize}

We've left a cell open for you below, but use as many as you need. Our
advice is to break the problem up into smaller parts you can run
separately. Check that each part is doing what you expect, then move on
to the next. You'll likely find that as you work through each part,
you'll need to go back and modify your previous code. This is totally
normal!

When training make sure you're updating only the weights of the
feed-forward network. You should be able to get the validation accuracy
above 70\% if you build everything right. Make sure to try different
hyperparameters (learning rate, units in the classifier, epochs, etc) to
find the best model. Save those hyperparameters to use as default values
in the next part of the project.

    \begin{Verbatim}[commandchars=\\\{\}]
{\color{incolor}In [{\color{incolor}7}]:} \PY{c+c1}{\PYZsh{} TODO: Build and train your network}
        \PY{c+c1}{\PYZsh{} from https://pytorch.org/tutorials/beginner/blitz/cifar10\PYZus{}tutorial.html\PYZsh{}training\PYZhy{}an\PYZhy{}image\PYZhy{}classifier}
        \PY{k}{class} \PY{n+nc}{Net}\PY{p}{(}\PY{n}{nn}\PY{o}{.}\PY{n}{Module}\PY{p}{)}\PY{p}{:}
            \PY{k}{def} \PY{n+nf}{\PYZus{}\PYZus{}init\PYZus{}\PYZus{}}\PY{p}{(}\PY{n+nb+bp}{self}\PY{p}{)}\PY{p}{:} 
                \PY{n+nb}{super}\PY{p}{(}\PY{n}{Net}\PY{p}{,} \PY{n+nb+bp}{self}\PY{p}{)}\PY{o}{.}\PY{n+nf+fm}{\PYZus{}\PYZus{}init\PYZus{}\PYZus{}}\PY{p}{(}\PY{p}{)}
                \PY{n+nb+bp}{self}\PY{o}{.}\PY{n}{conv1} \PY{o}{=} \PY{n}{nn}\PY{o}{.}\PY{n}{Conv2d}\PY{p}{(}\PY{l+m+mi}{3}\PY{p}{,} \PY{l+m+mi}{6}\PY{p}{,} \PY{l+m+mi}{5}\PY{p}{)}
                \PY{n+nb+bp}{self}\PY{o}{.}\PY{n}{pool} \PY{o}{=} \PY{n}{nn}\PY{o}{.}\PY{n}{MaxPool2d}\PY{p}{(}\PY{l+m+mi}{2}\PY{p}{,} \PY{l+m+mi}{2}\PY{p}{)}
                \PY{n+nb+bp}{self}\PY{o}{.}\PY{n}{conv2} \PY{o}{=} \PY{n}{nn}\PY{o}{.}\PY{n}{Conv2d}\PY{p}{(}\PY{l+m+mi}{6}\PY{p}{,} \PY{l+m+mi}{102} \PY{o}{*} \PY{l+m+mi}{2}\PY{p}{,} \PY{l+m+mi}{5}\PY{p}{)}
                \PY{n+nb+bp}{self}\PY{o}{.}\PY{n}{fc1} \PY{o}{=} \PY{n}{nn}\PY{o}{.}\PY{n}{Linear}\PY{p}{(}\PY{l+m+mi}{205} \PY{o}{*} \PY{l+m+mi}{102}\PY{p}{,} \PY{l+m+mi}{258} \PY{p}{)}
                \PY{n+nb+bp}{self}\PY{o}{.}\PY{n}{fc2} \PY{o}{=} \PY{n}{nn}\PY{o}{.}\PY{n}{Linear}\PY{p}{(}\PY{l+m+mi}{258}\PY{p}{,} \PY{l+m+mi}{200}\PY{p}{)}
                \PY{n+nb+bp}{self}\PY{o}{.}\PY{n}{fc3} \PY{o}{=} \PY{n}{nn}\PY{o}{.}\PY{n}{Linear}\PY{p}{(}\PY{l+m+mi}{200}\PY{p}{,} \PY{l+m+mi}{120}\PY{p}{)}
                
            \PY{k}{def} \PY{n+nf}{forward}\PY{p}{(}\PY{n+nb+bp}{self}\PY{p}{,} \PY{n}{x}\PY{p}{)}\PY{p}{:}
                \PY{n}{x} \PY{o}{=} \PY{n+nb+bp}{self}\PY{o}{.}\PY{n}{pool}\PY{p}{(}\PY{n}{F}\PY{o}{.}\PY{n}{relu}\PY{p}{(}\PY{n+nb+bp}{self}\PY{o}{.}\PY{n}{conv1}\PY{p}{(}\PY{n}{x}\PY{p}{)}\PY{p}{)}\PY{p}{)}
                \PY{n}{x} \PY{o}{=} \PY{n+nb+bp}{self}\PY{o}{.}\PY{n}{pool}\PY{p}{(}\PY{n}{F}\PY{o}{.}\PY{n}{relu}\PY{p}{(}\PY{n+nb+bp}{self}\PY{o}{.}\PY{n}{conv2}\PY{p}{(}\PY{n}{x}\PY{p}{)}\PY{p}{)}\PY{p}{)}
                \PY{n}{x} \PY{o}{=} \PY{n}{x}\PY{o}{.}\PY{n}{reshape}\PY{p}{(}\PY{o}{\PYZhy{}}\PY{l+m+mi}{1}\PY{p}{,} \PY{l+m+mi}{102} \PY{o}{*} \PY{l+m+mi}{205}\PY{p}{)}
                \PY{n}{x} \PY{o}{=} \PY{n}{F}\PY{o}{.}\PY{n}{relu}\PY{p}{(}\PY{n+nb+bp}{self}\PY{o}{.}\PY{n}{fc1}\PY{p}{(}\PY{n}{x}\PY{p}{)}\PY{p}{)}
                \PY{n}{x} \PY{o}{=} \PY{n}{F}\PY{o}{.}\PY{n}{relu}\PY{p}{(}\PY{n+nb+bp}{self}\PY{o}{.}\PY{n}{fc2}\PY{p}{(}\PY{n}{x}\PY{p}{)}\PY{p}{)}
                \PY{n}{x} \PY{o}{=} \PY{n+nb+bp}{self}\PY{o}{.}\PY{n}{fc3}\PY{p}{(}\PY{n}{x}\PY{p}{)}
                \PY{k}{return} \PY{n}{x}
        
        \PY{n}{model} \PY{o}{=} \PY{n}{Net}\PY{p}{(}\PY{p}{)}
        
        
        \PY{n}{criterion} \PY{o}{=} \PY{n}{nn}\PY{o}{.}\PY{n}{CrossEntropyLoss}\PY{p}{(}\PY{p}{)}
        \PY{n}{optimizer} \PY{o}{=} \PY{n}{optim}\PY{o}{.}\PY{n}{SGD}\PY{p}{(}\PY{n}{model}\PY{o}{.}\PY{n}{parameters}\PY{p}{(}\PY{p}{)}\PY{p}{,} \PY{n}{lr}\PY{o}{=}\PY{l+m+mf}{0.001}\PY{p}{,} \PY{n}{momentum}\PY{o}{=}\PY{l+m+mf}{0.9}\PY{p}{)}
\end{Verbatim}


    \subsection{Save the checkpoint}\label{save-the-checkpoint}

Now that your network is trained, save the model so you can load it
later for making predictions. You probably want to save other things
such as the mapping of classes to indices which you get from one of the
image datasets:
\texttt{image\_datasets{[}\textquotesingle{}train\textquotesingle{}{]}.class\_to\_idx}.
You can attach this to the model as an attribute which makes inference
easier later on.

\texttt{model.class\_to\_idx\ =\ image\_datasets{[}\textquotesingle{}train\textquotesingle{}{]}.class\_to\_idx}

Remember that you'll want to completely rebuild the model later so you
can use it for inference. Make sure to include any information you need
in the checkpoint. If you want to load the model and keep training,
you'll want to save the number of epochs as well as the optimizer state,
\texttt{optimizer.state\_dict}. You'll likely want to use this trained
model in the next part of the project, so best to save it now.

    \begin{Verbatim}[commandchars=\\\{\}]
{\color{incolor}In [{\color{incolor}9}]:} \PY{c+c1}{\PYZsh{} TODO: Save the checkpoint}
        \PY{c+c1}{\PYZsh{}model.checkpoint0}
\end{Verbatim}


    \subsection{Loading the checkpoint}\label{loading-the-checkpoint}

At this point it's good to write a function that can load a checkpoint
and rebuild the model. That way you can come back to this project and
keep working on it without having to retrain the network.

    \begin{Verbatim}[commandchars=\\\{\}]
{\color{incolor}In [{\color{incolor}10}]:} \PY{c+c1}{\PYZsh{} TODO: Write a function that loads a checkpoint and rebuilds the model}
         \PY{k}{def} \PY{n+nf}{load\PYZus{}checkpoint}\PY{p}{(}\PY{n}{filepath}\PY{p}{)}\PY{p}{:} 
             \PY{n}{checkpoint} \PY{o}{=} \PY{n}{torch}\PY{o}{.}\PY{n}{load}\PY{p}{(}\PY{n}{filepath}\PY{p}{)}
             \PY{n}{model} \PY{o}{=} \PY{n}{model}\PY{o}{.}\PY{n}{Network}\PY{p}{(}\PY{n}{checkpoint}\PY{p}{[}\PY{l+s+s1}{\PYZsq{}}\PY{l+s+s1}{input\PYZus{}size}\PY{l+s+s1}{\PYZsq{}}\PY{p}{]}\PY{p}{,} 
                                   \PY{n}{checkpoint}\PY{p}{[}\PY{l+s+s1}{\PYZsq{}}\PY{l+s+s1}{output\PYZus{}size}\PY{l+s+s1}{\PYZsq{}}\PY{p}{]}\PY{p}{,} 
                                   \PY{n}{checkpoint}\PY{p}{[}\PY{l+s+s1}{\PYZsq{}}\PY{l+s+s1}{hidden\PYZus{}layers}\PY{l+s+s1}{\PYZsq{}}\PY{p}{]}\PY{p}{)}
             \PY{n}{model}\PY{o}{.}\PY{n}{load\PYZus{}state\PYZus{}dict}\PY{p}{(}\PY{n}{model}\PY{o}{.}\PY{n}{state\PYZus{}dict}\PY{p}{(}\PY{p}{)}\PY{p}{)}
             
             \PY{k}{return} \PY{n}{model}
\end{Verbatim}


    \section{Inference for
classification}\label{inference-for-classification}

Now you'll write a function to use a trained network for inference. That
is, you'll pass an image into the network and predict the class of the
flower in the image. Write a function called \texttt{predict} that takes
an image and a model, then returns the top \(K\) most likely classes
along with the probabilities. It should look like

\begin{Shaded}
\begin{Highlighting}[]
\NormalTok{probs, classes }\OperatorTok{=}\NormalTok{ predict(image_path, model)}
\BuiltInTok{print}\NormalTok{(probs)}
\BuiltInTok{print}\NormalTok{(classes)}
\OperatorTok{>}\NormalTok{ [ }\FloatTok{0.01558163}  \FloatTok{0.01541934}  \FloatTok{0.01452626}  \FloatTok{0.01443549}  \FloatTok{0.01407339}\NormalTok{]}
\OperatorTok{>}\NormalTok{ [}\StringTok{'70'}\NormalTok{, }\StringTok{'3'}\NormalTok{, }\StringTok{'45'}\NormalTok{, }\StringTok{'62'}\NormalTok{, }\StringTok{'55'}\NormalTok{]}
\end{Highlighting}
\end{Shaded}

First you'll need to handle processing the input image such that it can
be used in your network.

\subsection{Image Preprocessing}\label{image-preprocessing}

You'll want to use \texttt{PIL} to load the image
(\href{https://pillow.readthedocs.io/en/latest/reference/Image.html}{documentation}).
It's best to write a function that preprocesses the image so it can be
used as input for the model. This function should process the images in
the same manner used for training.

First, resize the images where the shortest side is 256 pixels, keeping
the aspect ratio. This can be done with the
\href{http://pillow.readthedocs.io/en/3.1.x/reference/Image.html\#PIL.Image.Image.thumbnail}{\texttt{thumbnail}}
or
\href{http://pillow.readthedocs.io/en/3.1.x/reference/Image.html\#PIL.Image.Image.thumbnail}{\texttt{resize}}
methods. Then you'll need to crop out the center 224x224 portion of the
image.

Color channels of images are typically encoded as integers 0-255, but
the model expected floats 0-1. You'll need to convert the values. It's
easiest with a Numpy array, which you can get from a PIL image like so
\texttt{np\_image\ =\ np.array(pil\_image)}.

As before, the network expects the images to be normalized in a specific
way. For the means, it's \texttt{{[}0.485,\ 0.456,\ 0.406{]}} and for
the standard deviations \texttt{{[}0.229,\ 0.224,\ 0.225{]}}. You'll
want to subtract the means from each color channel, then divide by the
standard deviation.

And finally, PyTorch expects the color channel to be the first dimension
but it's the third dimension in the PIL image and Numpy array. You can
reorder dimensions using
\href{https://docs.scipy.org/doc/numpy-1.13.0/reference/generated/numpy.ndarray.transpose.html}{\texttt{ndarray.transpose}}.
The color channel needs to be first and retain the order of the other
two dimensions.

    \begin{Verbatim}[commandchars=\\\{\}]
{\color{incolor}In [{\color{incolor}12}]:} \PY{k}{def} \PY{n+nf}{process\PYZus{}image}\PY{p}{(}\PY{n}{image}\PY{p}{)}\PY{p}{:}
             \PY{l+s+sd}{\PYZsq{}\PYZsq{}\PYZsq{} Scales, crops, and normalizes a PIL image for a PyTorch model,}
         \PY{l+s+sd}{        returns an Numpy array}
         \PY{l+s+sd}{    \PYZsq{}\PYZsq{}\PYZsq{}}
             \PY{c+c1}{\PYZsh{}image = Image.open(image)}
             \PY{n}{img} \PY{o}{=} \PY{n}{Image}\PY{o}{.}\PY{n}{open}\PY{p}{(}\PY{n}{image}\PY{p}{)}
             \PY{n}{img\PYZus{}loader} \PY{o}{=} \PY{n}{transforms}\PY{o}{.}\PY{n}{Compose}\PY{p}{(}\PY{p}{[}\PY{n}{transforms}\PY{o}{.}\PY{n}{Resize}\PY{p}{(}\PY{l+m+mi}{256}\PY{p}{)}\PY{p}{,} 
                                             \PY{n}{transforms}\PY{o}{.}\PY{n}{CenterCrop}\PY{p}{(}\PY{l+m+mi}{224}\PY{p}{)}\PY{p}{,} 
                                             \PY{n}{transforms}\PY{o}{.}\PY{n}{ToTensor}\PY{p}{(}\PY{p}{)}\PY{p}{,} 
                                             \PY{n}{transforms}\PY{o}{.}\PY{n}{Normalize}\PY{p}{(}\PY{p}{[}\PY{l+m+mf}{0.485}\PY{p}{,} \PY{l+m+mf}{0.456}\PY{p}{,} \PY{l+m+mf}{0.406}\PY{p}{]}\PY{p}{,} \PY{p}{[}\PY{l+m+mf}{0.229}\PY{p}{,} \PY{l+m+mf}{0.224}\PY{p}{,} \PY{l+m+mf}{0.225}\PY{p}{]}\PY{p}{)}\PY{p}{]}\PY{p}{)}
             \PY{n}{img} \PY{o}{=} \PY{n}{img\PYZus{}loader}\PY{p}{(}\PY{n}{img}\PY{p}{)}
             \PY{n}{img} \PY{o}{=} \PY{n}{np}\PY{o}{.}\PY{n}{array}\PY{p}{(}\PY{n}{img}\PY{p}{)}
             \PY{k}{return} \PY{n}{img}
         
         \PY{c+c1}{\PYZsh{}train\PYZus{}data = process\PYZus{}image(trainloader.dataset)}
         
         \PY{k}{for} \PY{n}{epoch} \PY{o+ow}{in} \PY{n+nb}{range}\PY{p}{(}\PY{l+m+mi}{2}\PY{p}{)}\PY{p}{:} 
         
             \PY{n}{running\PYZus{}loss} \PY{o}{=} \PY{l+m+mf}{0.0}
             \PY{k}{for} \PY{n}{i}\PY{p}{,} \PY{n}{data} \PY{o+ow}{in} \PY{n+nb}{enumerate}\PY{p}{(}\PY{n}{process\PYZus{}image}\PY{p}{(}\PY{n}{trainloader}\PY{o}{.}\PY{n}{dataset}\PY{p}{)}\PY{p}{,}\PY{l+m+mi}{0}\PY{p}{)}\PY{p}{:}
                 \PY{n}{inputs}\PY{p}{,} \PY{n}{labels} \PY{o}{=} \PY{n}{data}
            
                 \PY{n}{optimizer}\PY{o}{.}\PY{n}{zero\PYZus{}grad}\PY{p}{(}\PY{p}{)}
                 
                 \PY{n}{outputs} \PY{o}{=} \PY{n}{net}\PY{p}{(}\PY{n}{inputs}\PY{p}{)}
                 \PY{n}{loss} \PY{o}{=} \PY{n}{criterion}\PY{p}{(}\PY{n}{outputs}\PY{p}{,} \PY{n}{labels}\PY{p}{)}
                 \PY{n}{loss}\PY{o}{.}\PY{n}{backward}\PY{p}{(}\PY{p}{)}
                 \PY{n}{optimizer}\PY{o}{.}\PY{n}{step}\PY{p}{(}\PY{p}{)}
         
                 \PY{n}{running\PYZus{}loss} \PY{o}{+}\PY{o}{=} \PY{n}{loss}\PY{o}{.}\PY{n}{item}\PY{p}{(}\PY{p}{)}
                 \PY{k}{if} \PY{n}{i} \PY{o}{\PYZpc{}} \PY{l+m+mi}{2000} \PY{o}{==} \PY{l+m+mi}{1999}\PY{p}{:}
                     \PY{n+nb}{print}\PY{p}{(}\PY{l+s+s1}{\PYZsq{}}\PY{l+s+s1}{[}\PY{l+s+si}{\PYZpc{}d}\PY{l+s+s1}{, }\PY{l+s+si}{\PYZpc{}5d}\PY{l+s+s1}{] loss: }\PY{l+s+si}{\PYZpc{}.3f}\PY{l+s+s1}{\PYZsq{}} \PY{o}{\PYZpc{}}
                           \PY{p}{(}\PY{n}{epoch} \PY{o}{+} \PY{l+m+mi}{1}\PY{p}{,} \PY{n}{i} \PY{o}{+} \PY{l+m+mi}{1}\PY{p}{,} \PY{n}{running\PYZus{}loss} \PY{o}{/} \PY{l+m+mi}{2000}\PY{p}{)}\PY{p}{)}
                     \PY{n}{running\PYZus{}loss} \PY{o}{=} \PY{l+m+mf}{0.0}
         
         \PY{n+nb}{print}\PY{p}{(}\PY{l+s+s1}{\PYZsq{}}\PY{l+s+s1}{Finished Training}\PY{l+s+s1}{\PYZsq{}}\PY{p}{)}
\end{Verbatim}


    \begin{Verbatim}[commandchars=\\\{\}]

        ---------------------------------------------------------------------------

        AttributeError                            Traceback (most recent call last)

        \textasciitilde{}\textbackslash{}Anaconda3\textbackslash{}lib\textbackslash{}site-packages\textbackslash{}PIL\textbackslash{}Image.py in open(fp, mode)
       2546     try:
    -> 2547         fp.seek(0)
       2548     except (AttributeError, io.UnsupportedOperation):
    

        AttributeError: 'ImageFolder' object has no attribute 'seek'

        
    During handling of the above exception, another exception occurred:
    

        AttributeError                            Traceback (most recent call last)

        <ipython-input-12-c7d2547c3452> in <module>()
         18 
         19     running\_loss = 0.0
    ---> 20     for i, data in enumerate(process\_image(trainloader.dataset),0):
         21         inputs, labels = data
         22 
    

        <ipython-input-12-c7d2547c3452> in process\_image(image)
          4     '''
          5     \#image = Image.open(image)
    ----> 6     img = Image.open(image)
          7     img\_loader = transforms.Compose([transforms.Resize(256), 
          8                                     transforms.CenterCrop(224),
    

        \textasciitilde{}\textbackslash{}Anaconda3\textbackslash{}lib\textbackslash{}site-packages\textbackslash{}PIL\textbackslash{}Image.py in open(fp, mode)
       2547         fp.seek(0)
       2548     except (AttributeError, io.UnsupportedOperation):
    -> 2549         fp = io.BytesIO(fp.read())
       2550         exclusive\_fp = True
       2551 
    

        AttributeError: 'ImageFolder' object has no attribute 'read'

    \end{Verbatim}

    To check your work, the function below converts a PyTorch tensor and
displays it in the notebook. If your \texttt{process\_image} function
works, running the output through this function should return the
original image (except for the cropped out portions).

    \begin{Verbatim}[commandchars=\\\{\}]
{\color{incolor}In [{\color{incolor} }]:} \PY{k}{def} \PY{n+nf}{imshow}\PY{p}{(}\PY{n}{image}\PY{p}{,} \PY{n}{ax}\PY{o}{=}\PY{k+kc}{None}\PY{p}{,} \PY{n}{title}\PY{o}{=}\PY{k+kc}{None}\PY{p}{)}\PY{p}{:}
            \PY{l+s+sd}{\PYZdq{}\PYZdq{}\PYZdq{}Imshow for Tensor.\PYZdq{}\PYZdq{}\PYZdq{}}
            \PY{k}{if} \PY{n}{ax} \PY{o+ow}{is} \PY{k+kc}{None}\PY{p}{:}
                \PY{n}{fig}\PY{p}{,} \PY{n}{ax} \PY{o}{=} \PY{n}{plt}\PY{o}{.}\PY{n}{subplots}\PY{p}{(}\PY{p}{)}
            
            \PY{c+c1}{\PYZsh{} PyTorch tensors assume the color channel is the first dimension}
            \PY{c+c1}{\PYZsh{} but matplotlib assumes is the third dimension}
            \PY{n}{image} \PY{o}{=} \PY{n}{image}\PY{o}{.}\PY{n}{numpy}\PY{p}{(}\PY{p}{)}\PY{o}{.}\PY{n}{transpose}\PY{p}{(}\PY{p}{(}\PY{l+m+mi}{1}\PY{p}{,} \PY{l+m+mi}{2}\PY{p}{,} \PY{l+m+mi}{0}\PY{p}{)}\PY{p}{)}
            
            \PY{c+c1}{\PYZsh{} Undo preprocessing}
            \PY{n}{mean} \PY{o}{=} \PY{n}{np}\PY{o}{.}\PY{n}{array}\PY{p}{(}\PY{p}{[}\PY{l+m+mf}{0.485}\PY{p}{,} \PY{l+m+mf}{0.456}\PY{p}{,} \PY{l+m+mf}{0.406}\PY{p}{]}\PY{p}{)}
            \PY{n}{std} \PY{o}{=} \PY{n}{np}\PY{o}{.}\PY{n}{array}\PY{p}{(}\PY{p}{[}\PY{l+m+mf}{0.229}\PY{p}{,} \PY{l+m+mf}{0.224}\PY{p}{,} \PY{l+m+mf}{0.225}\PY{p}{]}\PY{p}{)}
            \PY{n}{image} \PY{o}{=} \PY{n}{std} \PY{o}{*} \PY{n}{image} \PY{o}{+} \PY{n}{mean}
            
            \PY{c+c1}{\PYZsh{} Image needs to be clipped between 0 and 1 or it looks like noise when displayed}
            \PY{n}{image} \PY{o}{=} \PY{n}{np}\PY{o}{.}\PY{n}{clip}\PY{p}{(}\PY{n}{image}\PY{p}{,} \PY{l+m+mi}{0}\PY{p}{,} \PY{l+m+mi}{1}\PY{p}{)}
            
            \PY{n}{ax}\PY{o}{.}\PY{n}{imshow}\PY{p}{(}\PY{n}{image}\PY{p}{)}
            
            \PY{k}{return} \PY{n}{ax}
\end{Verbatim}


    \subsection{Class Prediction}\label{class-prediction}

Once you can get images in the correct format, it's time to write a
function for making predictions with your model. A common practice is to
predict the top 5 or so (usually called top-\(K\)) most probable
classes. You'll want to calculate the class probabilities then find the
\(K\) largest values.

To get the top \(K\) largest values in a tensor use
\href{http://pytorch.org/docs/master/torch.html\#torch.topk}{\texttt{x.topk(k)}}.
This method returns both the highest \texttt{k} probabilities and the
indices of those probabilities corresponding to the classes. You need to
convert from these indices to the actual class labels using
\texttt{class\_to\_idx} which hopefully you added to the model or from
an \texttt{ImageFolder} you used to load the data
(Section \ref{save-the-checkpoint}). Make sure to invert the dictionary
so you get a mapping from index to class as well.

Again, this method should take a path to an image and a model
checkpoint, then return the probabilities and classes.

\begin{Shaded}
\begin{Highlighting}[]
\NormalTok{probs, classes }\OperatorTok{=}\NormalTok{ predict(image_path, model)}
\BuiltInTok{print}\NormalTok{(probs)}
\BuiltInTok{print}\NormalTok{(classes)}
\OperatorTok{>}\NormalTok{ [ }\FloatTok{0.01558163}  \FloatTok{0.01541934}  \FloatTok{0.01452626}  \FloatTok{0.01443549}  \FloatTok{0.01407339}\NormalTok{]}
\OperatorTok{>}\NormalTok{ [}\StringTok{'70'}\NormalTok{, }\StringTok{'3'}\NormalTok{, }\StringTok{'45'}\NormalTok{, }\StringTok{'62'}\NormalTok{, }\StringTok{'55'}\NormalTok{]}
\end{Highlighting}
\end{Shaded}

    \begin{Verbatim}[commandchars=\\\{\}]
{\color{incolor}In [{\color{incolor} }]:} \PY{k}{def} \PY{n+nf}{predict}\PY{p}{(}\PY{n}{image\PYZus{}path}\PY{p}{,} \PY{n}{model}\PY{p}{,} \PY{n}{topk}\PY{o}{=}\PY{l+m+mi}{5}\PY{p}{)}\PY{p}{:}
            \PY{l+s+sd}{\PYZsq{}\PYZsq{}\PYZsq{} Predict the class (or classes) of an image using a trained deep learning model.}
        \PY{l+s+sd}{    \PYZsq{}\PYZsq{}\PYZsq{}}
            \PY{n}{image} \PY{o}{=} \PY{n}{process\PYZus{}image}\PY{p}{(}\PY{n}{image\PYZus{}path}\PY{p}{)}
            \PY{n}{tensor} \PY{o}{=} \PY{n}{torchvision}\PY{o}{.}\PY{n}{transforms}\PY{o}{.}\PY{n}{ToTensor}\PY{p}{(}\PY{p}{)}
            \PY{n}{image} \PY{o}{=} \PY{n}{tensor}\PY{p}{(}\PY{n}{image}\PY{p}{)}
            \PY{n}{image} \PY{o}{=} \PY{n}{image}\PY{o}{.}\PY{n}{unsqueeze\PYZus{}}\PY{p}{(}\PY{l+m+mi}{0}\PY{p}{)}
            
            \PY{k}{with} \PY{n}{torch\PYZus{}no\PYZus{}grad}\PY{p}{(}\PY{p}{)}\PY{p}{:} 
                \PY{n}{output} \PY{o}{=} \PY{n}{model}\PY{o}{.}\PY{n}{forward}\PY{p}{(}\PY{n}{image}\PY{o}{.}\PY{n}{cuda}\PY{p}{(}\PY{p}{)}\PY{p}{)}
                \PY{n}{probabililty} \PY{o}{=} \PY{n}{F}\PY{o}{.}\PY{n}{softmax}\PY{p}{(}\PY{n}{output}\PY{p}{,} \PY{n}{dim}\PY{o}{=}\PY{l+m+mi}{1}\PY{p}{)}
                \PY{n+nb}{print}\PY{p}{(}\PY{n}{probabililty}\PY{p}{)}
                
            
            \PY{k}{return} \PY{n}{probabililty}\PY{o}{.}\PY{n}{topk}\PY{p}{(}\PY{n}{topk}\PY{p}{)}
            \PY{c+c1}{\PYZsh{} TODO: Implement the code to predict the class from an image file}
\end{Verbatim}


    \subsection{Sanity Checking}\label{sanity-checking}

Now that you can use a trained model for predictions, check to make sure
it makes sense. Even if the validation accuracy is high, it's always
good to check that there aren't obvious bugs. Use \texttt{matplotlib} to
plot the probabilities for the top 5 classes as a bar graph, along with
the input image. It should look like this:

You can convert from the class integer encoding to actual flower names
with the \texttt{cat\_to\_name.json} file (should have been loaded
earlier in the notebook). To show a PyTorch tensor as an image, use the
\texttt{imshow} function defined above.

    \begin{Verbatim}[commandchars=\\\{\}]
{\color{incolor}In [{\color{incolor} }]:} \PY{c+c1}{\PYZsh{} TODO: Display an image along with the top 5 classes}
\end{Verbatim}



    % Add a bibliography block to the postdoc
    
    
    
    \end{document}
